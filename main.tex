\documentclass[12pt,aspectratio=169]{beamer}
\usefonttheme[onlymath]{serif}
\usepackage{thaispec}
\usepackage{graphicx}
\usepackage{colortbl}
\usepackage{tikz}
\usepackage{xcolor}
\usepackage{qrcode}

\usetheme{Copenhagen}
\usecolortheme{crane}

\begin{document}

\title{การประมาณค่าดัชนีคุณภาพอากาศ ณ \\จุดที่ไม่มีสถานีวัด \\ (Better Estimation of Absent Air Quality Indices) } 
\date{อาจารย์ที่ปรึกษา: ดร.รัฐพรหม พรหมคำ}
\author{นางสาวหฤทัย ชาหอม รหัสนักศึกษา 116210901007-1 \\
นางสาวเบญจมาศ สินแก้ว รหัสนักศึกษา 116210901010-5 \\ 
นายกิตติคุณ ปริญญาประเสริฐ รหัสนักศึกษา 116210901021-2}

\maketitle

\begin{frame}
    \frametitle{Problem}
    \begin{tikzpicture}[overlay, remember picture]
        \node at (7,-0.8) {\includegraphics[width=1.25\textwidth]{img/missing-station-aqi.png}};
    \end{tikzpicture}
    \end{frame}

    \begin{frame}
        \frametitle{การประมาณค่าเชิงเส้นคู่แบบทางเลือก}
        \begin{columns}
            \begin{column}{0.5\textwidth}
            กำหนดให้ $f(x,y)$ คือ ค่า AQI ที่พิกัด $(x,y)$:
            \begin{block}{}
              \[
                    f(x,y)=a+bx+cy+dxy
               \]   
            \end{block}

               \begin{center}
                   \includegraphics[width=\textwidth]{img/region-selection-one-point.pdf}
               \end{center}
            \end{column}
            \begin{column}{0.4\textwidth}  
            เมื่อ $a,b,c$ และ $d$ คือผลเฉลยของสมการ $AX = B$  โดยที่              
            \[
               A
               =\begin{bmatrix}
               1 & x_1 & y_1 & x_1y_1\\
               1 & x_2 & y_2 & x_2y_2\\
               1 & x_3 & y_3 & x_3y_3\\
               1 & x_4 & y_4 & x_4y_4\\
               \end{bmatrix},
        \]
        \[
               X = \begin{bmatrix}
                  a \\ b \\ c \\ d
               \end{bmatrix}
               ,
               B
               =\begin{bmatrix}
                  f(x_1,y_1) \\
                  f(x_2,y_2) \\
                  f(x_3,y_3) \\
                  f(x_4,y_4)
               \end{bmatrix}
        \]
            \end{column}
            \end{columns}
\end{frame}
    
\begin{frame}
\frametitle{การเเบ่งกั้นบริเวณรอบจุดเป้าหมาย}
\begin{center}
                \begin{figure}
                    \includegraphics[scale=0.5]{img/NEP_t.pdf}
                \end{figure}
               $ NE(P_t) = \{(x_{i} , y_{i}) \in \mathbb{R}^2 | x_{i} > x_{t} \text{ และ } y_{i} > y_{t}\} $
            \end{center}
\end{frame}

\begin{frame}
\frametitle{การเเบ่งกั้นบริเวณรอบจุดเป้าหมาย}
\begin{center}
                \begin{figure}
                    \includegraphics[scale=0.5]{img/NWP_t.pdf}
                \end{figure}
               $NW(P_t) = \{(x_{i} , y_{i}) \in \mathbb{R}^2 | x_{i} < x_{t} \text{ และ } y_{i} > y_{t}\}$
            \end{center}
\end{frame}

\begin{frame}
\frametitle{การเเบ่งกั้นบริเวณรอบจุดเป้าหมาย}
\begin{center}
                \begin{figure}
                    \includegraphics[scale=0.5]{img/SWP_t.pdf}
                \end{figure}
                $SW(P_t) = \{(x_{i} , y_{i}) \in \mathbb{R}^2 | x_{i} < x_{t} \text{ และ } y_{i} < y_{t}\} $
            \end{center}
\end{frame}

\begin{frame}
\frametitle{การเเบ่งกั้นบริเวณรอบจุดเป้าหมาย}
\begin{center}
        \begin{figure}
                    \includegraphics[scale=0.5]{img/SEP_t.pdf}
                \end{figure}
                $SE(P_t) = \{(x_{i} , y_{i}) \in \mathbb{R}^2 | x_{i} > x_{t} \text{ และ } y_{i} < y_{t}\}$
            \end{center}
\end{frame}

\begin{frame}
\frametitle{การเลือกจุดที่ล้อมรอบบริเวณเป้าหมาย}
    \begin{center}
        \begin{figure}
            \includegraphics[scale=0.5]{img/selection.pdf}
        \end{figure}
    \end{center}
\end{frame}

\begin{frame}
    \frametitle{การพิจารณาเงื่อนไขจุดยอดของสี่เหลี่ยม }
\begin{theorem} 
	กำหนดให้ $P_i=(x_i,y_i) \in \mathbb{R}^2$ \text{และ}  $m_{ij} =\left(\dfrac{y_j-y_i}{x_j-x_i} \right)$ \text{ และ } $c_{ij} = y-m_{ij}x $ สำหรับ $i,j = 1,2,3,4$ และ $i \neq  j$ 
ถ้าเงื่อนไขต่อไปนี้เป็นจริง 
	\begin{enumerate}
\item \label{condition1} $x_{1}  \neq x_{2} \neq x_{3} \neq x_{4}$
\item \label{condition2} $y_3 - m_{12}x_3 + c_{12} \neq 0$
\item \label{condition3} $y_4 - m_{12}x_4 + c_{12} \neq 0$
\item \label{condition4} $y_4 - m_{23}x_4 + c_{23} \neq 0$
\item \label{condition5} $y_4 - m_{13}x_4 + c_{13} \neq 0$
\end{enumerate}
แล้ว $P_1,P_2,P_3$ และ $P_4$ เป็นจุดยอดของรูปสี่เหลี่ยม
\end{theorem}
\end{frame}

\begin{frame}{ตัวอย่าง}
    กำหนดให้  \[P_t = (1.75387,0.24043), \]
    \begin{center}
        \begin{tabular}{lccc} 
        \hline
       $P_i$ & ($x_i$,$y_i$) &  $f(x_i,y_i)$ \\
        \hline
        $P_1$ &$(1.75479,0.24063)$ &$74$\\
        $P_2$ &$(1.75516,0.24119)$ &$65$\\
        $P_3$ &$(1.75378,0.24084)$ &$63$\\
        $P_4$ &$(1.75296,0.24054)$ &$78$\\
        $P_5$ &$(1.75319,0.23996)$ &$78$\\
        $P_6$ &$(1.75292,0.23946)$ &$72$\\
        $P_7$ &$(1.75403,0.24010)$ &$89$\\
        $P_8$ &$(1.75396,0.23972)$ &$96$\\
        $P_9$ &$(1.75399,0.23909)$ &$87$\\
        $P_{10}$ &$(1.75570,0.24037)$ &$72$\\
        \hline
        \end{tabular}
    \end{center}
\end{frame}

\begin{frame}{ตัวอย่าง}
        \begin{center}
            \begin{tabular}{lccc} 
            \hline
           $P_i$ & ($x_i$,$y_i$) &  บริเวณรอบจุดเป้าหมาย & $d(P_t,P_i)$\\
            \hline
            \cellcolor{orange!30}$P_1$ &\cellcolor{orange!30}$(1.75479,0.24063)$ &\cellcolor{orange!30}$NE(P_t)$  &\cellcolor{orange!30}$9.37E-4$\\
            $P_2$ &$(1.75516,0.24119)$ &$NE(P_t)$&$1.49E-3$\\
            \hline
            \cellcolor{orange!30}$P_3$ &\cellcolor{orange!30}$(1.75378,0.24084)$ &\cellcolor{orange!30}$NW(P_t)$ &\cellcolor{orange!30}$4.18E-4$\\
            $P_4$ &$(1.75296,0.24054)$ &$NW(P_t)$&$9.23E-4$\\
            \hline
            \cellcolor{orange!30}$P_5$ &\cellcolor{orange!30}$(1.75319,0.23996)$ &\cellcolor{orange!30}$SW(P_t)$&\cellcolor{orange!30}$8.25E-4$\\
            $P_6$ &$(1.75292,0.23946)$ &$SW(P_t)$&$1.36E-3$\\
            \hline
            \cellcolor{orange!30}$P_7$ &\cellcolor{orange!30}$(1.75403,0.24010)$ &\cellcolor{orange!30}$SE(P_t)$&\cellcolor{orange!30}$3.74E-4$\\
            $P_8$ &$(1.75396,0.23972)$ &$SE(P_t)$&$7.15E-4$\\
            $P_9$ &$(1.75399,0.23909)$ &$SE(P_t)$&$1.34E-3$\\
            $P_{10}$ &$(1.75570,0.24037)$ &$SE(P_t)$&$1.83E-3$\\
            \hline
            \end{tabular}
        \end{center}
\end{frame}

\begin{frame}{ตัวอย่าง}
พิจารณา $P_t = (x_t,y_t) =  (1.75387, 0.24043)$, $P_i = (x_i, y_i)$ โดยที่
\begin{columns}
\column{0.45\textwidth}
\begin{align*}
P_1 & = (1.75479, 0.24063) \in NE(P_t), \\ 
P_3 & = (1.75378, 0.24084) \in NW(P_t),\\
P_5 & = (1.75319, 0.23996) \in SW(P_t),\\
P_7 & = (1.75403, 0.24010)  \in SE(P_t), 
\end{align*}
\[m_{ij} = \dfrac{y_j-y_i}{x_j-x_i}, \]
\[c_{ij} = y_i - m_{ij}x_i,\]    
\column{0.45\textwidth} 
\begin{block}{}\
\textbf{จะได้ว่า} 
\begin{enumerate}
    \item $x_1 \neq x_3 \neq x_5 \neq x_7$
    \item $y_5 - m_{13}x_5 - c_{13} \neq 0$
    \item $y_7 - m_{13}x_7 - c_{13} \neq 0$
    \item $y_7 - m_{35}x_7 - c_{35} \neq 0$
    \item $y_7 - m_{15}x_7 - c_{15} \neq 0$
\end{enumerate}     
\end{block} 
\end{columns}


\end{frame}

\begin{frame}{ตัวอย่าง}
        \begin{columns}
            \begin{column}{0.6\textwidth}
\begin{align*}
f(P_1) & = f(x_1,y_1) = f(1.75479, 0.24063) = 74, \\ 
f(P_3) & = f(x_3,y_3) = f(1.75378, 0.24084) = 63,\\
f(P_5) & = f(x_5,y_5) = f(1.75319, 0.23996) = 78,\\
f(P_7) & = f(x_7,y_7) = f(1.75403, 0.24010) = 89, 
\end{align*}
จะได้ว่า
%\vspace{-7mm}
            \begin{block}{}
            \vspace{-3mm}
              \begin{align*}
                 f(P_t) & = f(x_t,y_t) \\
                        & = f(1.75387, 0.24043)\\
                        & = a+bx_t+cy_t+dx_ty_t \\
                        & \approx 75.83
              \end{align*}  
            \end{block}

            \end{column}
            \begin{column}{0.35\textwidth} 

{\color{orange!90!black}%
เมื่อ $a,b,c$ และ $d$ คือผลเฉลยของสมการ $AX = B$  โดยที่              
            \[
               A
               =\begin{bmatrix}
               1 & x_1 & y_1 & x_1y_1\\
               1 & x_3 & y_3 & x_3y_3\\
               1 & x_5 & y_5 & x_5y_5\\
               1 & x_7 & y_7 & x_7y_7\\
               \end{bmatrix},
        \]
        \[
               X = \begin{bmatrix}
                  a \\ b \\ c \\ d
               \end{bmatrix}
               ,
               B
               =\begin{bmatrix}
                  f(x_1,y_1) \\
                  f(x_3,y_3) \\
                  f(x_5,y_5) \\
                  f(x_7,y_7)
               \end{bmatrix}
        \]
}
            \end{column}

            \end{columns}    
\end{frame}

\begin{frame}
    \frametitle{ค่าคลาดเคลื่อน}

    \begin{tikzpicture}[overlay, remember picture]
        \node at (7,-2.8) {\includegraphics[width=1.25\textwidth]{img/error-test.jpg}};
    \end{tikzpicture}    
    \end{frame}

\begin{frame}{ค่าคลาดเคลื่อน}

    \begin{tikzpicture}[overlay, remember picture]
        \node at (7,-2.8) {\includegraphics[width=1.25\textwidth]{img/error-test.jpg}};
        \node[fill=white, rounded corners=0.1cm] (p) at (9,0) {\Huge \texttt{117}};
        \draw[->, white, line width=1.5mm] (p) -- (6.1,-0.8);
    \end{tikzpicture}    
    \end{frame}


\begin{frame}{ค่าคลาดเคลื่อน}
        \begin{center}
        \begin{tabular}{lcccccc} 
        \hline
        $P_i$ & $(x_i,y_i)$&AQI & $f(x_i,y_i)$ &  $E_{abs}$ & $\varepsilon_t$\\
        \hline
        1 & (13.90451806 , 100.5277991)   & 152 & 139.46   &12.54 &8.25\% \\
        2 & (13.77523348 , 100.5695343)  &124 & 117.01   &6.99  &5.64\% \\
        3 & (13.72971342 , 100.5365753) &127   & 137.5   &10.5  &8.27\% \\
        4  & (13.76206192 , 100.5506516)  &122  & 108.35   &13.65 &11.19\% \\
        5 & (13.77640054 , 100.5697083) &129 & 104.56   &24.44 &18.95\% \\
        6  & (13.05272355 , 101.1030755) &127  & 121.15   &5.85  &4.61\% \\
        7& (13.72954666 , 100.5363191) &152 & 131.48   &20.52 &13.50\% \\
        8 &  (13.72988017 , 100.5364923) &152 & 138.13   &13.87 &9.13\% \\
        9 & (18.25021998 , 99.76271963) &89 & 104.83   &15.83 &17.79\% \\
        10 & (17.16178591 , 104.1341291) &97 & 150.97   &53.97 &55.64\% \\
        \hline
        \end{tabular}
    \end{center}
\end{frame}

\begin{frame}
    \frametitle{Web application}
        \begin{center}
            \begin{figure}
                \centering
                \includegraphics[width=\textwidth]{img/web-app.jpeg}
            \end{figure}
        \end{center}
    \end{frame}

    \begin{frame}{Use case}
            \begin{center}
                \begin{figure}
                    \centering
                    \includegraphics[width=\textwidth]{img/web-app-click.jpeg}
                \end{figure}
            \end{center}
        \end{frame}    

\begin{frame}{Tech Stack}
\frametitle{Tech Stack}
    \begin{center}
        \begin{figure}
            \centering
            \includegraphics[width=\textwidth]{img/tech-stack.pdf}
        \end{figure}
    \end{center}
\end{frame}

\end{document}